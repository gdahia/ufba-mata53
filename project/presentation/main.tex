\documentclass[14pt]{beamer}

\usepackage[utf8]{inputenc}

\usepackage{amsmath}
\usepackage{amsthm}
\DeclareMathOperator*{\argmin}{arg\,min}

\title{Minimum branching spanning tree}
\subtitle{Trabalho prático - MATA53}
\author{Gabriel Dahia}
\institute{Universidade Federal da Bahia}
\date{2017}

\begin{document}

\frame{\titlepage}

\begin{frame}
\frametitle{Definições}

\begin{itemize}
\item $G = (V, E)$: simples, conexo e não-direcionado;
\item $T \in \mathcal{T}(G)$: árvore geradora de $G$;
\item \emph{Branch}: vértice com grau superior a 2;
\item $b(T)$: número de \emph{branches} em $T$.
\end{itemize}

\end{frame}

\begin{frame}
\frametitle{Problema MinBST}

Dado $G = (V, E)$, queremos $T^*$ tal que:
$$T^* = \argmin_{T \in \mathcal{T}(G)}(b(T))$$

\end{frame}

\begin{frame}
\frametitle{Problema MinBST}

  \begin{center}
    \includegraphics<1>[width=0.75\textwidth]{example/example-01.png}
    \includegraphics<2>[width=0.75\textwidth]{example/example-02.png}
    \includegraphics<3>[width=0.75\textwidth]{example/example-03.png}
    \includegraphics<4>[width=0.75\textwidth]{example/example-04.png}
  \end{center}

\end{frame}

\begin{frame}
  \frametitle{Aplicações práticas}
  Redes ópticas:
  \begin{itemize}
    \item Comunicação \emph{multicast};
    \item Multiplexação (DWDM).
  \end{itemize}
\end{frame}

\begin{frame}
  \frametitle{Problemas relacionados}
  \begin{itemize}
    \item Caminho Hamiltoniano;
    \item Minimizar folhas -- maximizar vértices internos;
    \item \emph{Minimum connnected dominating set};
    \item \emph{Maximum path node problem}.
  \end{itemize}
\end{frame}

\begin{frame}
  \frametitle{Abordagens}
  \begin{itemize}
    \item Problema NP-Díficil;
    \item Programação inteira e heurísticas;
    \item Garantias?
    \item \emph{Algoritmos aproximativos}.
  \end{itemize}
\end{frame}

\end{document}
