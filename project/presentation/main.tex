\documentclass[14pt]{beamer}

\usepackage[utf8]{inputenc}
\usetheme{Berlin}

\usepackage{amsmath}
\usepackage{amsthm}
\DeclareMathOperator*{\argmin}{arg\,min}

\title{Minimum branching spanning tree}
\subtitle{Trabalho prático - MATA53}
\author{Gabriel Dahia}
\institute{Universidade Federal da Bahia}
\date{2017}

\begin{document}

\frame{\titlepage}

\begin{frame}
\frametitle{Definições}
  \begin{itemize}
    \item $G = (V, E)$: simples, conexo e não-direcionado, $n$ vértices e $m$ arestas;
    \item $T \in \mathcal{T}(G)$: árvore geradora de $G$;
    \item \emph{Branch}: vértice com grau superior a 2;
    \item $b(T)$: número de \emph{branches} em $T$.
  \end{itemize}
\end{frame}

\begin{frame}
\frametitle{Problema MinBST}
  \begin{block}{MinBST}
    Dado $G = (V, E)$, queremos $T^*$ tal que:
    $$T^* = \argmin_{T \in \mathcal{T}(G)}(b(T))$$
  \end{block}
\end{frame}

\begin{frame}
\frametitle{Problema MinBST}
  \begin{center}
    \includegraphics<1>[width=0.75\textwidth]{example/example-01.png}
    \includegraphics<2>[width=0.75\textwidth]{example/example-02.png}
    \includegraphics<3>[width=0.75\textwidth]{example/example-03.png}
    \includegraphics<4>[width=0.75\textwidth]{example/example-04.png}
  \end{center}
\end{frame}

\begin{frame}
  \frametitle{Aplicações práticas}
  Redes ópticas:
  \begin{itemize}
    \item Comunicação \emph{multicast};
    \item Multiplexação (DWDM).
  \end{itemize}
\end{frame}

\begin{frame}
  \frametitle{Problemas relacionados}
  \begin{itemize}
    \item Caminho Hamiltoniano;
    \item Minimizar folhas -- maximizar vértices internos;
    \item \emph{Maximum path node problem};
    \item \emph{Minimum connnected dominating set}.
  \end{itemize}
\end{frame}

\begin{frame}
  \frametitle{Abordagens}
  \begin{itemize}
    \item Problema NP-Difícil;
    \item Programação inteira e heurísticas;
    \item Formulação complementar;
    \item Algoritmos exatos?
    \item Aproximativos?
  \end{itemize}
\end{frame}

\begin{frame}
  \frametitle{Aproximabilidade}
  \begin{itemize}
    \item Valor ótimo igual a 0;
    \item Redução de \emph{Minimum set cover}: MinBST~$\not\in$~APX;
    \item Se P $\ne$ NP, então melhor aproximação é $\Omega(\log n)$.
  \end{itemize}
\end{frame}

\begin{frame}
  \frametitle{Algoritmo aproximativo}
  \begin{itemize}
    \item $G$ uniformemente denso: $\delta(G) \ge cn$, $c \in [0, 1)$;
    \item Algoritmo guloso;
    \item $b(T) \le 3\lceil \log_{\frac{1}{1 - c}}n\rceil - 1$;
    \item $b(T) \in O(\log n)$;
    \item Se $G$ não admite caminho hamiltoniano, melhor aproximação.
  \end{itemize}
\end{frame}

\begin{frame}
  \frametitle{Algoritmo aproximativo}
  \begin{center}
    \includegraphics<1>[width=0.75\textwidth]{example/salamon/salamon-01.png}
    \includegraphics<2>[width=0.75\textwidth]{example/salamon/salamon-02.png}
    \includegraphics<3>[width=0.75\textwidth]{example/salamon/salamon-03.png}
    \includegraphics<4>[width=0.75\textwidth]{example/salamon/salamon-04.png}
  \end{center}
\end{frame}

\begin{frame}
  \frametitle{Algoritmo aproximativo}
  Lemas:
  \begin{enumerate}
    \item $i$-ésimo passo, existe vértice com $ca_i$ vizinhos brancos;
    \item $O(1)$ \emph{branches} por passo;
    \item $O(\log n)$ passos no total;
    \item conecta componentes com $O(\log n)$.
  \end{enumerate}
\end{frame}

\begin{frame}
  \frametitle{Resultado mais geral?}
  \begin{itemize}
    \item<1> E se $\frac{m}{n} \ge cn$, $c \in [0, \frac{1}{2})$?
    \item<1> Se $\frac{|G[A \cup B]| - |G[B]|}{|A \cup B|} \ge c|A|$, então vale o Lema 1;
    \item<1> Qual vértice escolher?
    \item<1,3,5> Escolha maximal viola Lema 1;
    \item<1,3,5> Escolha minimal viola Lema 3.
    \item<2,3> E se $G$ não possuir $K_3$?
    \item<3> Não.
    \item<4-> E se escolher dois de uma vez?
    \item<5> Também não.
  \end{itemize}
\end{frame}

\begin{frame}
  \frametitle{Conclusões}
  \begin{itemize}
    \item MinBST é NP-Difícil;
    \item Aproximação restrita;
    \item Exatos e aproximativos desconhecidos, mesmo quando restrito a classes;
    \item Afetado pela estrutura geral do grafo.
  \end{itemize}
\end{frame}

\end{document}
