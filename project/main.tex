\documentclass[12pt]{article}

\usepackage{sbc-template}

\usepackage{graphicx,url}

\usepackage[utf8]{inputenc} 

\sloppy

\title{A preliminary report for the minimum branch vertices problem}

\author{Gabriel Dahia\inst{1}}

\address{Universidade Federal da Bahia \email{gdahia@gmail.com}}

\begin{document} 

\maketitle

\begin{abstract}
  A known generalization of the Hamiltonian path problem, the problem of \emph{minimum branch vertices} is to find a spanning tree of a given graph $G$ with the minimum number of vertices with degree greater than 2.
  It is already estabilished that this problem is not in \emph{APX}; an approximation preserving reduction from the \emph{minimum set cover} problem shows that, unless \emph{P = NP}, it is not approximable better than a multiplicative ratio of $\Omega(\log n )$ for a graph on $n$ vertices.
  In this paper, we show, through a combinatorial upper-bound, that an approximation algorithm for another degree-based spanning tree optimization problem is also an approximation for \emph{minimum branch vertices} when the graph is non-traceable.
  In fact, it is the best known approximation for general non-traceable graphs and, based on the assumption that \emph{P $\neq$ NP}, it is the best possible for some classes of graphs.
\end{abstract}
     
\section{Introduction}

While the first reasons for studying degree-based spanning tree optimization problems were their practical applications in the design of communication networks, they are of theoretical interest for their close relation to the Hamiltonian path problem.
The \emph{minimum branch vertices problem}, specifically, asks for the spanning tree $T$ of a given graph $G$ such that the number of vertices with degree higher than 2 in $T$ is minimum.
If the graph is \emph{traceable}, \emph{i.e.} if it admits a Hamiltonian path, the minimum number of branch vertices (vertices with degree at least 3) in any spanning tree is 0; that spanning tree is itself a Hamiltonian path.

\cite{gargano2002, gargano2004}, who first considered the minimum number of branch vertices in any spanning tree, showed this problem to be NP-Hard and, for any fixed $k$, they showed that it is NP-Complete to decide if a graph admits a spanning tree with at most $k$ branch vertices. 
They also related the minimum number of branch vertices to other graph theoretic parameters and gave a polynomial-time algorithm for finding a spanning tree with at most one branch vertex (which they called a spanning spider), if the minimum degree-sum of any 3 element independent set is at least the number of vertices in the graph minus one.

\cite{flandrin2008} proved a new condition for the existence of a spanning spider: they showed that it is enough that there are two vertices whose degree-sum is the total number of vertices in the graph minus one.

\cite{salamon2010} proved, through an approximation preserving reduction from the minimum set cover problem, that, unless P = NP, the minimum branch vertices problem is not approximable better than a multiplicative ratio of $\Omega(\log n)$ for a graph on $n$ vertices.
They also provide the first approximation for minimum branch vertices: given a graph of order $n$ in which every vertex has degree at least $cn$, where $c$ is a real constant, their algorithm finds a spanning tree with at most $3\lceil \log_{\frac{1}{1 - c}}n\rceil + 1$ branch vertices in polynomial time.

\cite{chimani2015} consider the complementary formulation for minimum branch vertices: maximizing the number of vertices whose degree in the spanning tree is at most 2.
They call it \emph{maximum path-node} problem and say that it has better approximability than minimum branch vertices. 
For instance, every spanning tree is a 1/2-approximation for maximum path node.
Their contribution is a 6/11-approximation algorithm and a proof that the maximum path-node problem is APX-Hard.

\bibliographystyle{sbc}
\bibliography{bib}

\end{document}
