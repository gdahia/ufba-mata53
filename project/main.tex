\documentclass[12pt]{article}

\usepackage{sbc-template}

\usepackage{graphicx,url}

\usepackage[utf8]{inputenc} 

\usepackage{amsmath}

\sloppy

\DeclareMathOperator*{\argmin}{arg\,min}

\title{A preliminary report for the minimum branch vertices problem}

\author{Gabriel Dahia\inst{1}}

\address{Universidade Federal da Bahia \email{gdahia@gmail.com}}

\begin{document} 

\maketitle

\begin{abstract}
  A known generalization of the Hamiltonian path problem, the problem of \emph{minimum branch vertices} is to find a spanning tree of a given graph $G$ with the minimum number of vertices with degree greater than 2.
  It is already estabilished that this problem is not in \emph{APX}; an approximation preserving reduction from the \emph{minimum set cover} problem shows that, unless \emph{P = NP}, it is not approximable better than a multiplicative ratio of $\Omega(\log n )$ for a graph on $n$ vertices.
  In this paper, we show, through a combinatorial upper-bound, that an approximation algorithm for another degree-based spanning tree optimization problem is also an approximation for \emph{minimum branch vertices} when the graph is non-traceable.
  In fact, it is the best known approximation for general non-traceable graphs and, based on the assumption that \emph{P $\neq$ NP}, it is the best possible for some classes of graphs.
\end{abstract}
     
\section{Introduction}

While the first reasons for studying degree-based spanning tree optimization problems were their practical applications in the design of communication networks, they are of theoretical interest for their close relation to the Hamiltonian path problem.
The \emph{minimum branch vertices problem}, specifically, asks for the spanning tree $T$ of a given graph $G$ such that the number of vertices with degree higher than 2 in $T$ is minimum.
If the graph is \emph{traceable}, \emph{i.e.} if it admits a Hamiltonian path, the minimum number of branch vertices (vertices with degree at least 3) in any spanning tree is 0; that spanning tree is itself a Hamiltonian path.

\cite{gargano2002, gargano2004}, who first considered the minimum number of branch vertices, showed this problem to be NP-Hard and, for any fixed $k$, they showed that it is NP-Complete to decide if a graph admits a spanning tree with at most $k$ branch vertices. 
They also related the minimum number of branch vertices to other graph theoretic parameters and gave a polynomial-time algorithm for finding a spanning tree with at most one branch vertex (which they called a spanning spider), if the minimum degree-sum of any 3 element independent set is at least the number of vertices in the graph minus one.

\cite{flandrin2008} extended that work, proving a new condition for the existence of a spanning spider in a graph: they showed that it is enough that there are two vertices whose degree-sum is the order of the graph minus one.

\cite{salamon2010} proved, through an approximation preserving reduction from the minimum set cover problem, that, unless P = NP, the minimum branch vertices problem is not approximable better than a multiplicative ratio of $\Omega(\log n)$ for a graph on $n$ vertices.
They also provide the first approximation for minimum branch vertices: given a graph of order $n$ in which every vertex has degree at least $cn$, where $c$ is a real constant, their algorithm finds a spanning tree with at most $3\lceil \log_{\frac{1}{1 - c}}n\rceil + 1$ branch vertices in polynomial time.

\cite{chimani2015} consider the complementary formulation for minimum branch vertices: maximizing the number of vertices whose degree in the spanning tree is at most 2.
They call it \emph{maximum path-node} problem and say that it has better approximability than minimum branch vertices. 
For instance, every spanning tree is a 1/2-approximation for maximum path node.
Their contribution is a 6/11-approximation algorithm and a proof that the maximum path-node problem is APX-Hard.

In order to solve the minimum branch vertices problem in practice, approaches based on heuristics and integer programming have been proposed.
\cite{cerulli2009} related minimum branch vertices to the problem of finding a spanning tree that minimizes the degree sum of its branch vertices; they also showed that the problems are not equivalent and provided a single formulation to solve them.
\cite{cerrone2014} considered the relation of these two problems to minimizing the number of leaves (vertices whose degree is exactly one) and proposed an evolutionary algorithm that attempted to solve them.
Recent work on heuristics and integer programming has been able to solve multiple instances to optimality -- we refer the reader to \cite{marin2015, melo2016, silvestri2017} for detailed discussions and compared results.

%todo: address other degree based opt probs

In this work, we will attempt to provide approximations for the minimum branch vertices problem for general graphs and claw-free graphs.

\section{Notations, basic definitions and problem statement}

We denote a simple, connected and undirected graph by $G = (V, E)$.
The set of spanning trees of $G$ is given by $\mathcal{T}(G)$, and unless noted otherwise, $T$ is always a spanning tree of G, that is, $T \in \mathcal{T}(G)$.
The degree of a vertex $v$ in a graph $G$ is denoted $d_G(v)$.
$\delta(G) = \min_{v \in V}(d_G(v))$ is the minimum degree of a vertex in $G$ and $\Delta(G) = \max_{v \in V}(d_G(v))$ is the maximum.
We use the notation in \cite{furer1992} for $\Delta^* = \min_{T \in \mathcal{T}(G)}(\Delta(T))$, the minimum degree of all spanning trees of $G$, and the notation in \cite{gargano2004} for $s(G) = \min_{T \in \mathcal{T}(G)}(|\{v \in V \mid d_T(v) > 2 \}|)$, the minimum number of branch vertices in $G$.

A graph $G$ is \emph{$H$-free} if $H$ is not an induced subgraph of $G$.
The standard notation for a complete bipartite graph with partition sizes $s$ and $t$ is $K_{s, t}$; we adopt that notation throughout this work.
Furthermore, $K_{1, 3}$ has a special denomintation for its importance in the study of hamiltonicity -- it is called a \emph{claw}.
We will use \emph{claw-free} instead of $K_{1,3}$\emph{-free} for legibility.

Formally, the minimum branch vertices problem is, given $G$, find a spanning tree $T^*$ in which $|\{v \in V \mid d_T(v) > 2 \}|) = s(G)$, or:
$T^* = \argmin_{T \in \mathcal{T}(G)}(|\{ v \in V \mid d_T(v) > 2\}|)$.

\section{Methodology}

\bibliographystyle{sbc}
\bibliography{bib}

\end{document}
