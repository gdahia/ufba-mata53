\documentclass[12pt]{article}

\usepackage{sbc-template}

\usepackage{graphicx,url}

\usepackage[utf8]{inputenc} 

\usepackage{amsmath}
\usepackage{amsthm}

\sloppy

\newtheorem{theorem}{Theorem}[section]
\newtheorem{fact}[theorem]{Fact}
\newtheorem{corollary}{Corollary}[theorem]
\newtheorem{lemma}[theorem]{Lemma}

\DeclareMathOperator*{\argmin}{arg\,min}

\title{On finding spanning trees with few branches}

\author{Gabriel Dahia\inst{1}}

\address{Federal University of Bahia \email{gdahia@gmail.com}}

\begin{document} 

\maketitle

\section{Approximation}

\begin{itemize}
  \item Rule 1: If $T$ contains a vertex $v$ with only one white neighbor , expand $v$;
  \item Rule 2: If $T$ contains a vertex $v$ with only two white neighbors, one of which (call it $w$) would remain with at least 3 white neighbors in after the expasion of $v$, then expand $v$ and then expand $w$;
  \item Rule 3: If $T$ contains a vertex $v$ with three or more white neighbors, expand $v$;
  \item Rule 4: If $T$ contains two vertices $v$ and $u$, each with only two and the same white neighbors, assign $v$ one neighbor and $u$ the other, arbitrarily;
  \item Rule 5: If $T$ contains a vertex $v$ with only two white neighbors, expand $v$.
    If this produces a vertex $w$ in $T$ with only one white neighbor, expand $w$; otherwise, mark $v$ blue;
  \item Rule 6: If $T$ is empty, expand the highest degree node of $V(G)$.
\end{itemize}

\begin{lemma}
  Let $T$ be the spanning tree of $G$ constructed by Algorithm 1 and let $B(T)$ be the set of vertices marked blue in $T$.
  Then
  $$ P(T) \ge \frac{3(V(T) - B(T))}{5} $$
\end{lemma}
\begin{proof}
  For any subtree $T$ of $G$ constructed during the algorithm, define the \emph{potential} of $T$ as:
  $$ \mathcal{P}(T) = \frac{5|P(T)|}{3} - (V(T) - B(T)) $$
  We argue that $\mathcal{P}(T) \ge 0$ throughout the algorithm.
  Clearly, $\mathcal{P}(T) \ge 0$ when $T$ is empty.
  Now, suppose some expansion rule was applied to $T$, yielding $T'$.
  It is enough to show that $\Delta \mathcal{P} = \mathcal{P}(T') - \mathcal{P}(T) \ge 0$.

  Rule 6 is only applied once -- let $v$ be the vertex to which it is applied.
  Since $d_G(v) = \Delta(G)$, $|P(T)|$ increases by $\Delta(G)$, $|V(T)|$ increases by $\Delta(G) + 1$ and $|B(T)|$ remains unchanged.
  We then have, since we assumed $\Delta(G) \ge 3$,
  \begin{align*}
    \Delta \mathcal{P} &= \frac{5\Delta(G)}{3} - (\Delta(G) + 1) \\
                       &= \frac{5\Delta(G) - 3\Delta(G) - 3}{3} \\
                       &= \frac{2\Delta(G) - 3}{3} \\
                       &\ge \frac{6 - 3}{3} \\
    \Delta \mathcal{P} &\ge 1
  \end{align*}

  If Rule 1 is applied, $|P(T)|$ and $|V(T)|$ both increase by one and $|B(T)|$ remains unchanged.
  Then, we have $\Delta \mathcal{P} = \frac{5}{3} - 1 = \frac{2}{3} \ge 0$.

  If Rule 2 is applied to vertices $v$ and $w$, assuming $w$ has $x \ge 3$ white neighbors after the expansion of $v$, $|P(T)|$ increases by $x$ (as neither $v$ nor $w$ remain path nodes), $|V(T)|$ increases by $2 + x$ and $|B(T)|$ remains unchanged.
  We have, then,
  \begin{align*}
    \Delta \mathcal{P} &= \frac{5x}{3} - (x + 2) \\
                       &= \frac{5x - 3x - 6}{3}\\
                       &= \frac{2(x - 3)}{3} \\
    \Delta \mathcal{P} &\ge 0
  \end{align*}

  Applying Rule 3 to a vertex $v$ with $x \ge 3$ neighbors increases $|P(T)|$ by $x - 1$ (note that $v$ will no longer be a path node), $|V(T)|$ by $x$ and leaves $|B(T)|$ unchanged.
  Therefore,
  \begin{align*}
    \Delta \mathcal{P} &= \frac{5(x - 1)}{3} - x \\
                       &= \frac{5x - 5 - 3x}{3} \\
                       &= \frac{2x - 5}{3} \\
                       &\ge \frac{6 - 5}{3} \\
    \Delta \mathcal{P} &\ge \frac{1}{3}
  \end{align*}

  If Rule 4 is applied, both $|P(T)|$ and $|V(T)|$ increase by 2 and $|B(T)|$ remains the same.
  Then, $\Delta \mathcal{P} = \frac{10}{3} - 2 = \frac{4}{3} \ge 0$.

  We divide Rule 5 in two cases.
  The first case is when the expansion of $v$ produces a vertex $w$ in $T$ with only a single white neighbor.
  Expanding $v$ and then expanding $w$ increases $|P(T)|$ by 2, $|V(T)|$ by 3 and leaves $|B(T)|$ unchanged.
  Hence, $\Delta \mathcal{P}~=~\frac{10}{3}~-~3~=~\frac{1}{3}~\ge~0$.

  The second case, we expand $v$ and mark it blue.
  This increases both $|P(T)|$ and $|B(T)|$ by 1 and $|V(T)|$ by 2: $\Delta \mathcal{P} = \frac{5}{3} - (2 - 1) = \frac{2}{3} \ge 0$.

  Inductively, this shows that, when the algorithm produces a spanning tree $T$, $\mathcal{P}(T) = \frac{5}{3}|P(T)| - (V(T) - B(T)) \ge 0$.
  Thus,
  $$ P(T) \ge \frac{3(V(T) - B(T))}{5} $$

\end{proof}

\begin{lemma} \label{lemma:sep}
  For any $i > 0$, the set of gray vertices of $T_i$ is a separator of $G$.
  That is, as $T_i$ is built, there is no edge incident to both a black vertex and a white vertex.
\end{lemma}

\begin{proof}
  Suppose, by contradiction, that there is $u \in V(T_i)$ so that, as $T_i$ is built, $v$ is white and a neighbor of $u$.
  Since $u$ is black, it has either been expanded in some tree $T_j$, or it has been assigned one of its neighbors while its other neighbor was assigned to some vertex that is also black in $T_j$, $j \le i$.
  If $u$ was expanded, by definition of the expansion rule, all of its neighbors are colored gray.
  Similarly, both neighbors of $u$ are gray when it was colored black by applying Rule 4.
  Vertex $v$, then, is colored gray as $T_j$ is built.
  It is clear by the algorithm's definition that once a vertex is colored gray, its color never returns to white.
  Therefore, $v$ cannot be both white and a neighbor of $u$.
\end{proof}

\begin{lemma}
  A vertex has no neighbor in a higher layer.
\end{lemma}

\begin{proof}
  % mostrar que a prioridade das regras (primeiro pegar o q tem 3, depois oq tem vizinhos total sobrepostos e depois sobreposicao simples), nessa ordem, impede vizinhos em camadas superiores
  Suppose, by contradiction, that a vertex $v$ has some neighbors in a higher layer.
  Let $i$ be the smallest index of a higher layer for which $v$ has a neighbor and let $u$ be the $i$th-blue marked vertex.
  When $u$ is marked blue, vertex $v$ cannot have been expanded or assigned, for this would make its neighbors reside in layers equal or lower than its own.
  We divide the remainder of the proof according to the supposed number of neighbors $v$ has in some higher layer.

  Vertex $v$ cannot have 3 or more neighbors in a higher layer, for if it did, Rule 3 could have been applied before $u$ was marked blue, a contradiction.
  Therefore, $v$ can have at most 2 neighbors in higher layers.

  Since $v$ has not been expanded or assigned and $v$ has a vertex in a layer $i$, by Lemma~\ref{lemma:sep} $v$ is gray when $u$ is marked blue.
  The number of white neighbors of $v$ immediately before $u$ is marked blue is then exactly two, otherwise $v$ would have been expanded instead of $u$.
  It is also impossible for the expansion of $u$ to color one of $v$'s white neighbors grey: in this case, the aplication of Rule 4 or Rule 5 without marking $u$ blue would have been preferred.
  Hence, a white neighbor of $v$ can only be colored grey as a result of the application of Rules 1-4 and Rule 5 without marking another vertex blue.

\end{proof}

\end{document}
